\documentclass[11pt]{article}
\usepackage{fullpage} % better margins
\usepackage{amsthm,amssymb,amsmath} % useful math commands
\usepackage{graphicx} % figures
\usepackage{xfrac} % inline fractions
\usepackage{enumerate} % custom enum symbols
\usepackage{algpseudocode, algorithm} % for pseudocode

% Define some useful shortcuts
\newcommand{\R}{\mathbb{R}}
\newcommand{\C}{\mathbb{C}}
\newcommand{\N}{\mathbb{N}}
\newcommand{\Z}{\mathbb{Z}}
\newcommand{\Q}{\mathbb{Q}}
\newcommand{\M}{\mathcal{M}}
\newcommand{\I}{\mathcal{I}}
\newcommand{\E}{\mathcal{E}}
\renewcommand{\O}{\mathcal{O}}
% if you want to use \alg and \opt in text, wrap them in math mode to make the spacing work
% e.g. We have shown that $\alg$ runs in $\O(n)$
\newcommand{\alg}{\textsc{alg}}
\newcommand{\opt}{\textsc{opt}}

\newtheorem{theorem}{Theorem}[section]
\newtheorem{corollary}{Corollary}[theorem]
\newtheorem{lemma}[theorem]{Lemma}

\title{Solutions for 611 Homework 2}

\author{Mahim Agarwal, Hanwen Xiong, and Jackson Warley}

\begin{document}

\maketitle

\section{Solution to Question 1}

Let $\M = (E, \I)$. For all $i \in [k]$, we have $|\emptyset \cap E_i| = 0 \leq 1$, so $\emptyset \in \I$.
To prove heredity, let $S \in \I$ and suppose $R \subseteq S$.
Then, if $i \in [k]$, we have $R \cap E_i \subseteq S \cap E_i$, implying $|R \cap E_i| \leq |S \cap E_i| \leq 1$.
To prove exchange, first note the following property: because the (nonempty) sets $E_1, \dots, E_k$ partition $E$ and each independent set intersects any $E_i$ in at most one element, any $S \in \I$ has nontrivial intersection with exactly $|S|$ of the sets $E_1, \dots, E_k$.
Suppose $A, B \in \I$ with $|A| < |B|$.
Then $B$ intersects $|B| > |A|$ of the sets $E_1, \dots, E_k$, so there must be some $i$ such that $E_i \cap A = \emptyset$ and $E_i \cap B = \{b\}$.
Then $A \cup \{b\} \in \I$, since $|(A \cup \{b\}) \cap E_i| = 1$, proving that $\M$ is a matroid.

\section{Solution to Question 2}

We will present the algorithm first, followed by the analysis.
\begin{algorithm}
\begin{algorithmic}
  \Function{alg}{$P = \{p_1, p_2, \dots, p_n\}$}
    \State $I \gets \{[p_1, p_1 + 1]\}$
    \For{$p \in P$ left to right}
      \State $[a, b] \gets \operatorname{last}(I)$
      \If{$p > b$}
      \State $I \gets I \cup \{[p, p + 1]\}$
      \EndIf
    \EndFor
  \EndFunction
\end{algorithmic}
\end{algorithm}

\noindent {\bf Claim 1:} $\alg$ runs in $\O(n)$ time and returns the smallest possible set of intervals covering all the points in $P$.
\begin{proof}
  Since $\alg$ terminates exactly when we have considered every point in $P$, and $P$ is finite, $\alg$ must terminate.
  When it does, $I$ will contain some number of intervals, greater than $1$.
  If $\alg$ ever encounters a point $p$ that is not covered by an interval already in $I$, it adds an interval containing $p$ to $I$.
  Thus, since every point is considered, all points are covered when $\alg$ terminates.

  Now, we must show that $P$ cannot be covered using fewer than $|I|$ (after $\alg$ has run to completion) unit-length intervals.
  Let $P' = \{p \in P \mid [p, p + 1] \in I\}$.
  Since every interval in $I$ begins with a point from $P$, $|P'| = |I|$.
  Furthermore, since $\alg$ only constructs a new interval when the nearest endpoint is further than $1$ away from the point under consideration, no two points in $P'$ are within $1$ of each other.
  Thus, any cover of $P'$ must include at least $|P'| = |I|$ intervals, and in particular the solution yielded by $\alg$ is optimal.

  Finally, $\alg$ considers each point in $P$ exactly once.
  The work of comparing the points and retrieving the most recently added interval requires only constant time, so considering a single point is $\O(1)$.
  Thus, $\alg$ has a total runtime of $\O(n)$.
\end{proof}




\section{Solution to Question 3}

\begin{enumerate}[\indent (1)]
    \item $(\Rightarrow)$ We prove the contrapositive.
      Let $S \subseteq J$ and suppose that there exists some $d \in [n]$ for which $S_d > d$.
      Then, by the pigeonhole principle, in any ordering of $S$ there must be some job $s_i$ with $i > d$ and $d_{s_i} < d$.
      Since we can idenfity an overdue job $s_i$ for any ordering of $S$, $S$ must not be good.

      $(\Leftarrow)$ Let $S \subseteq J$ with $S_d \leq d$ for all $d \in [n]$ and suppose toward a contradiction that $S$ is not good.
      Then in any ordering of $S$, there is some overdue job.
      Let $d + 1$ be the index of the first overdue job in $S$ and choose the ordering $S = \{s_1, \dots, s_k\}$ that maximizes $d$.
      Then $S$ is of the form $S = \{s_1, \dots, s_d, s_{d+1}, \dots, s_k\}$ where jobs $s_1$ through $s_{d}$ yield bonuses and $s_{d+1}$ does not.
      Suppose that for some $i \in [d]$, we have $d_{s_i} > d$.
      Then we could exchange $s_i$ with $s_{d+1}$, after which $s_{d+1}$ would yield a bonus, and, since $d_{s_i} > d$, so would $s_i$.
      However, this would contradict the maximality of $d$, since it would result in an ordering in which the first $d+1$ jobs yield bonuses.
      This means that for all $i \in [d]$, $d_{s_i} \leq d$, implying that $S_d \geq d$.
      But we also have $d_{s_{d+1}} \leq d$, since $s_{d+1}$ is overdue.
      Thus $S_d > d$, a contradiction, whereby $S$ must be good.

    \item To prove heredity, suppose that $A \in \I$.
      Let $A = \{a_1, \dots, a_k\}$ be a good ordering of $A$, and let $B \subseteq A$ be ordered identically up to removal of elements.
      Since $B$ is obtained by removing jobs from $A$, the index in $B$ of any job in $A \cap B$ is at most its index in $A$, so any job that yielded a bonus in $A$ still yields one in $B$.
      Thus, since $A$ was good, so is $B$, whence $B \in \I$ as desired.

      Now, to prove exchange, let $A, B \in \I$ with $|A| < |B|$.
      Suppose that $A = \{a_1, \dots, a_k\}$ and $B = \{b_1, \dots, b_j\}$.
      Let $A' = A \cup \{b_j\}$.
      Since $B \in \I$, we must have $d_{b_j} \geq j \geq k + 1$, therefore $b_j$ yields a bonus in $A'$.
      The other jobs in $A'$ have not moved from their positions in $A$, so they still yield bonuses.
      Thus $A' \in \I$, implying $(J, \I)$ has independence.

    \item Let $w(j) = b_j$ for all $j \in J$.
    Since $(J, \I)$ is a matroid, we know that we can apply the matroid greedy algorithm to find a maximum-weight independent set.
\end{enumerate}

\section{Solution to Question 4}



\section{Solution to Question 5}



\section{Solution to Question 6}



\end{document}
