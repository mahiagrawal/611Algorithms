\documentclass[11pt]{article}
\usepackage{fullpage} % better margins
\usepackage{amsthm,amssymb,amsmath} % useful math commands
\usepackage{graphicx} % figures
\usepackage{xfrac} % inline fractions
\usepackage{enumitem} % custom enum symbols

% Define some useful shortcuts
\newcommand{\R}{\mathbb{R}}
\newcommand{\C}{\mathbb{C}}
\newcommand{\N}{\mathbb{N}}
\newcommand{\Z}{\mathbb{Z}}
\newcommand{\Q}{\mathbb{Q}}
\newcommand{\M}{\mathcal{M}}
\newcommand{\I}{\mathcal{I}}
\newcommand{\E}{\mathcal{E}}
\renewcommand{\O}{\mathcal{O}}
% if you want to use \alg and \opt in text, wrap them in math mode to make the spacing work
% e.g. We have shown that $\alg$ runs in $\O(n)$
\newcommand{\alg}{\textsc{alg}}
\newcommand{\opt}{\textsc{opt}}

\newtheorem{theorem}{Theorem}[section]
\newtheorem{corollary}{Corollary}[theorem]
\newtheorem{lemma}[theorem]{Lemma}

\title{Solutions for 611 Homework 1}

\author{Mahim Agarwal, Hanwen Xiong, and Jackson Warley}

\begin{document}

\maketitle

\section{Solution to Question 1}



\section{Solution to Question 2}

For clarity, let the elements of $A$ and $B$ be indexed as $A = \{a_1, \dots, a_n\}$ and $B = \{b_1, \dots, b_n\}$.
Consider the following algorithm $\alg$ for determining the $n$th smallest element of $A \cup B$:
\begin{enumerate}
  \item Let $i = j = \frac{n}{2}$ be indices into $A$ and $B$, respectively, which may be updated in later steps.
  \item If it is the case that $a_i < b_{j+1}$ and $b_j < a_{i+1}$, return $\max \{a_i, b_j\}$.
  \item Otherwise, if $a_i > b_{j+1}$, update $i$ and $j$ so that $i \mapsto \frac{1}{2}i$ and $j \mapsto \frac{1}{2}(j + n)$.
    If, on the other hand, $b_j > a_{i+1}$, update the indices so that $i \mapsto \frac{1}{2}(i + n)$ and $j \mapsto \frac{1}{2}j$.
    If $i$ (resp. $j$) is equal to $n$, return $a_i$ (resp. $b_j$); otherwise, return to step $2$.
\end{enumerate}
We will use the following lemma to prove the correctness of $\alg$.

\begin{lemma}
  \label{nth-smallest-lemma}
  Suppose that $i + j = n$ and let $m = max \{a_i, b_j\}$. Then $m$ is the $n$th smallest element of $A \cup B$ if and only if $a_i < b_{j+1}$ and $b_j < a_{i+1}$.
\end{lemma}

\begin{proof}
  $(\Rightarrow)$ Suppose that $m$ is the $n$th smallest element of $A \cup B$.
  Then $m$ is larger than exactly $n - 1$ elements of $A \cup B$.
  Where can these elements possibly be located in their lists?
  If $m = a_i$, then $m$ is greater than exactly $i - 1$ elements of $A$ (since $A$ is sorted).
  Thus $m$ must be greater than exactly $(n - 1) - (i - 1) = (i + j - 1) - (i - 1) = j$ elements of $B$.
  In fact, since $B$ is sorted, $m$ must be greater than the first $j$ elements of $B$.
  In particular, this means we have $a_{i + 1} > a_i = m > b_j$, giving us one of the desired inequalities.
  Moreover, $m > b_{j+1}$ would imply the presence of an $n$th element of $A \cup B$ which is smaller than $m$, contradicting that $m$ is the $n$th smallest element of $A \cup B$.
  Thus, we have shown $a_i = m \leq b_{j+1}$, and, since the elements are distinct, this inequality must be strict, as desired.
  A symmetrical argument yields the same result in the case that $m = b_j$, completing one direction of the proof.

  $(\Leftarrow)$ Now suppose that $a_i < b_{j+1}$ and $b_j < a_{i+1}$.
  Since $m \geq a_i$, it is greater than at least $i - 1$ elements of $A$.
  Likewise, since $m \geq b_j$, it is greater than at least $j - 1$ elements of $B$.
  In fact, since $m = \max\{a_i, b_j\}$, one of these inequalities is strict and the other is actually an equality.
  Without loss of generality (the other possible choice is dealt with by a symmetrical argument), suppose $m = a_i$ and $m > b_j$.
  Then $m$ is greater than exactly $i - 1$ elements of $A$ and at least $j$ elements of $B$, so $m$ is greater than at lest $(i - 1) + j = n - 1$ elements of $A \cup B$.
  We initially supposed that $m = a_i < b_{j+1}$, so this lower bound is actually tight.
  Since $m$ is greater than exactly $n - 1$ elements of $A \cup B$, $m$ is the $n$th smallest element of $A \cup B$.
\end{proof}

\noindent {\bf Claim 1:} $\alg$ always terminates and returns the $n$th largest element of $A \cup B$.

\begin{proof}
  Firstly, note that initially we have $i = j = \frac{n}{2}$, so $i + j = n$.
  When we update the indices, we do so by $(i, j) \mapsto \left(\frac{1}{2}i, \frac{1}{2}(j + n)\right)$.
  Since \[\frac{1}{2}i + \frac{1}{2}(j + n) = \frac{1}{2}(i + j) + \frac{1}{2}n = n\] we have that $i + j = n$ is an invariant of $\alg$.
  This allows us to apply Lemma \ref{nth-smallest-lemma} at any time during our analysis of $\alg$.

  First, we argue that $\alg$ will always terminate.
  Need to show that trajectories of the indices are monotonic.
  This means we can't loop infinitely.
  Since we always return at extremal values, we must return every time.
  Finally, we are correct in the extremal case and we are correct if we return early by the lemma.
\end{proof}

\noindent {\bf Claim 2:} $\alg$ runs in $\O(\log n)$ time.

\begin{proof}
  The updating of $i$ and $j$ amounts to a binary search for the correct index configuration.
  Since the position of $j$ is completely determined by the position of $i$ (lest we violate the invariant $i + j = n$), it suffices to constrain our analysis to the trajectory of $i$ under iteration of the map $i \mapsto \frac{1}{2}i$.
  Each iteration halves the value of $i$, so there may be at most $\log_{2}n = \O(\log n)$ iterations before the algorithm terminates due to the extremal case of $i = 1$.
  Restricting our attention to the case where $i$ decreases rather than increases maintains generality since we could make the same argument about the trajectory of $j$ in the case that $i$ increases during a run of $\alg$.
\end{proof}
\section{Solution to Question 3}



\section{Solution to Question 4}



\section{Solution to Question 5}



\end{document}
