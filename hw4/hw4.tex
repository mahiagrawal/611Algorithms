\documentclass[11pt]{article}
\usepackage{fullpage} % better margins
\usepackage{amsthm,amssymb,amsmath} % useful math commands
\usepackage{graphicx} % figures
\usepackage{xfrac} % inline fractions
\usepackage{enumerate} % custom enum symbols
\usepackage{algpseudocode,algorithm} % for pseudocode

% Define some useful shortcuts
\newcommand{\R}{\mathbb{R}}
\newcommand{\C}{\mathbb{C}}
\newcommand{\N}{\mathbb{N}}
\newcommand{\Z}{\mathbb{Z}}
\newcommand{\Q}{\mathbb{Q}}
\newcommand{\M}{\mathcal{M}}
\newcommand{\I}{\mathcal{I}}
\newcommand{\E}{\mathcal{E}}
\renewcommand{\O}{\mathcal{O}}
% if you want to use \alg and \opt in text, wrap them in math mode to make the spacing work
% e.g. We have shown that $\alg$ runs in $\O(n)$
\newcommand{\alg}{\textsc{alg}}
\newcommand{\opt}{\textsc{opt}}

\newtheorem{theorem}{Theorem}[section]
\newtheorem{corollary}{Corollary}[theorem]
\newtheorem{lemma}[theorem]{Lemma}

\title{Solutions for 611 Homework 4}

\author{Mahim Agarwal, Hanwen Xiong, and Jackson Warley}

\begin{document}

\maketitle

\section{Solution to Question 1}

Define interted pair to be a pair $(a_i, a_j)$ such that $a_i > a_j$ while $i < j$.
Observe that each swap eliminates only one inverted pairs. 
Therefore, the number of swaps we should apply to a list is the number of all inverted pairs in this list.

\noindent We have to do at least no swap when the list is sorted and at most $\frac{n(n-1)}{2}$ swaps when the list is reversely sorted.
Let $X$ be the number of swaps we have to do in the randomly sorted list.
Then the expectation of swaps necessary before the list is in sorted order is

\begin{equation*}
	E(number\ of\ swaps) = \sum_{i = 0}^{\frac{n(n-1)}{2}} iP[X = i]
\end{equation*}

\noindent {\bf Claim: } $P[X = i] = P[X = \frac{n(n-1)}{2} - i]$

\begin{proof}
	Note that we can do $i$ swaps on a list that has $\frac{n(n-1)}{2} - i$ of inverted pairs to add anthoer $i$ inverted pairs to this list and therefore make it reversely sorted.
	That is, there're $i$ inverted pairs in the reversely sorted list.
	Since the sorted list and the reversely sorted list have the same length, the number of their corresponding lists which have $i$ inverted pairs are the same.
	Thus comes our claim.
\end{proof}

\noindent Using the claim above, we can add up terms like $iP[X = i]$ and $[\frac{n(n-1)}{2} - i]P[X = \frac{n(n-1)}{2} - i]$ and get something like:
\begin{equation*}
		2E(number\ of\ swaps) = \sum_{i = 0}^{\frac{n(n-1)}{2}} [\frac{n(n-1)}{2}]P[X = i] = \frac{n(n-1)}{2} \sum_{i = 0}^{\frac{n(n-1)}{2}} P[X = i] = \frac{n(n-1)}{2}
\end{equation*}

\noindent So, the final result is:
\begin{equation*}
		E(number\ of\ swaps) = \frac{n(n-1)}{4}
\end{equation*}

\newpage

\section{Solution to Question 2}

\textbf{Part 1:}

\textbf{Algorithm}
\begin{enumerate}
	\item Run the given algorithm $k$ number of times. $(k = 250)$
	\item Return the most frequent output in these k runs.
\end{enumerate}
To determine the minimum $k$ so that the correct answer is the most frequent output after k runs, consider the following:

The given algorithm gives correct answer with $P = 2/3$ and one of the $n$ other incorrect answers with $P = 1/3$.
Since nothing is specified for the probability distribution of incorrect answers, we will find the minimum number of trials needed for 
the worst case possible. 

\noindent {\bf Claim 1:} The minimum number of trials($k$) to get the correct answer most number of times will be maximum when the 
probability distribution of incorrect answers is such that the algorithm always returns one specific wrong answer whenever it returns
the incorrect answer.

\textbf{Please Note} that the worst case here is on the number of trials(k) required.

\begin{proof}
In $k$ runs, if the correct answer is returned $k/2+1$ number of times, then regardless of the probability distribution of incorrect 
answers, we will always get correct answer as the mode (and final output). Thus, the worst case would be when the correct 
answer is returned $k/2$ times but it still was not the mode. This can only happen when one particular incorrect answer appeared exactly 
$k/2$ number of times. \newline
Thus, the above claim holds true.
\end{proof}
Hence, the worst case on number of trials required is when the algorithm returns the correct answer with $P = 2/3$ while it returns a 
specific incorrect answer(let's say $x$) with $P = 1/3$. \newline
So to find minimum $k$ for worst case, the probability distribution of the given algorithm is binomial.

Taking analogy to the coin toss problem where heads can occur with probability p, the expected number of times we get incorrect answer
in k trials, $E(X) = k/3$, where $X$ is the random variable on number of times we see incorrect answer.

Consider the following random variable: \\
$X_i = \left \{
\begin{tabular}{cc}
1 & when algorithm returns incorrect answer i number of times \\
0 & otherwise
\end{tabular}
\right \}$ \\
So $X_1, X_2,.....,X_k$ are independent boolean random variables such that \\
for $X = \sum_{i} X_i$, $\mu = E[X] = k/3$ \\
Using chernoff bounds: \\
$P[X \geq (1+\delta)\mu] \leq e^{-\delta^2 \mu/3}$ \\
Substituting $\delta = 1/2$ and the value of $\mu$ \\
$P[X \geq k/2] \leq e^{-k/36}$ \\
$1 - P[X < k/2] \leq e^{-k/36}$ \\
$P[X < k/2] \geq 1 - e^{-k/36}$ \\
Now, $P(success) = P($we get the incorrect answer less than half times in k runs$)$ \\
$\Rightarrow P(success) = P(X < k/2)$ \\
$\Rightarrow P(success) \geq 1 - e^{-k/36} \geq 0.999$ \\
$\Rightarrow k > 36ln(1000)$ \\
So, minimum number of runs needed, $k = 250$

\textbf{Part 2:} When n is very large and each incorrect answer is equally likely. \\
Since n is very large and any incorrect answer have same probability to occur $(1/(3n))$ , any incorrect answer will have very
low probability to occur twice. Thus, we can run the algorithm until an output occurs twice and return that output to be the minimum 
vertex cover. \\
\textbf{Probability Analysis:} \\
Let $X$ be the random variable on number of times the output is correct in k trials.\\
So, the expected number of times we received the correct output, $E(X) = 2k/3$ l=similar to previous part.  \\
Consider the following random variable: \\
$X_i = \left \{
\begin{tabular}{cc}
1 & when algorithm returns correct answer i number of times \\
0 & otherwise
\end{tabular}
\right \}$ \\
So $X_1, X_2,.....,X_k$ are independent boolean random variables such that \\
for $X = \sum_{i} X_i$, $\mu = E[X] = 2k/3$ \\
Using chernoff bounds: \\
$P[X \leq (1-\delta)\mu] \leq e^{-\delta^2 \mu/2}$ \\
Substituting $\delta = 1-3/2k$ and the value of $\mu$ \\
$P[X \leq 1] \leq e^{-(2k-3)^2/6k}$ \\
$1 - P[X \leq 1] \leq e^{-(2k-3)^2/6k}$ \\
$P[X \geq 2] \geq 1 - e^{-(2k-3)^2/6k}$ \\
Now, $P(success) = P($we get the incorrect answer less than 2 times$)$ \\
$\Rightarrow P(success) = P(X \geq 2)$ \\
$\Rightarrow P(success) \geq 1 - e^{-(2k-3)^2/6k} \geq 0.999$ \\
$\Rightarrow k > 13.87$ \\
So, minimum number of runs needed, $k = 14$

\newpage

\section{Solution to Question 3}

\begin{enumerate}[(1)]
  \item Even a very simple algorithm will work with high probability:

    \begin{algorithm}
      \begin{algorithmic}
        \Function{alg}{$G$}
        \Loop
          \State pick $k$ vertices uniformly at random to form $X$
          \If{$|E(G[X])| \geq \frac{mk(k-1)}{n(n-1)}$}
            \State \Return $X$
          \EndIf
        \EndLoop
        \EndFunction
      \end{algorithmic}
    \end{algorithm}

    Because it always chooses $k$ vertices for $X$ and only terminates when the desired condition is met, $\alg$ is correct.
    We argue that its expected runtime is polynomial in $n$ and $m$.

    {\bf Claim 1:} If $X \subseteq V$ of size $k$ is selected uniformly at random, the expected number of edges in $G[X]$ is $\frac{mk(k-1)}{n(n-1)}$.

    \begin{proof}
      For each $(u,v) \in E(G)$, define a random variable \[X_{(u,v)} = \begin{cases}1 & u, v \in X\\ 0 & \textnormal{otherwise}.\end{cases}\]
      Then the expected number of edges in $G[X]$ is $E\left[\sum_{e \in E(G)}X_e\right]$.
      If $e = (u,v)$ then \[E[X_e] = P(u \in X) P(v \in X \mid u \in X) = \frac{k}{n}\frac{k-1}{n-1}.\]
      Thus, \[E\left[\sum_{e \in E(G)}X_e\right] = \sum_{e \in E(G)} E[X_e] = mE[x_e] = \frac{mk(k-1)}{n(n-1)}\] as desired.
    \end{proof}

    {\bf Claim 2:} $\alg$ has runtime polynomial in $m$ and $n$.

    \begin{proof}
      Choosing $X$ randomly can be done in multiple ways, but a na\"ive approach requires no more than $\Theta(n)$ time (e.g. by iterating through all the vertices and choosing them or not).
      Checking whether the candidate set $X$ satisfies the edge count requirement by iterating through the vertices in $X$ and counting edges incident to them incurs another $\Theta(m)$ time.
      Thus each iteration of the loop requires time linear in $m$ and $n$.
      The expected runtime of the algorithm is therefore $\Theta(m + n)$ times the expected number of times the loop runs.
      The latter quantity can be computed as follows.
      Let $X$ be a random variable equal to the number of edges in $G[X]$ on a given run.
      Let $p = P(X \geq E[X])$ be the probability of ``success.''
      Since $X$ is integer-valued, we have
      \begin{align*}
        E[X] &\leq (E[X] - 1)P(X \leq E[x] - 1) + Mp\\
        E[X] &\leq (E[X] - 1)(1 - p) + Mp\\
        E[X] &\leq (E[X] - 1) + (E[X] - 1)p + Mp\\
        1 &\leq Mp - (E[X] - 1)p\\
        \frac{1}{M - E[X] + 1} &\leq p
      \end{align*} where $M$ is the maximum value $X$ can take (note that dividing by $(M - E[X] + 1)$ is only okay because we know $M \geq E[X]$).
      Therefore the probability that $\alg$ finds a suitable $X$ on any given attempt is \[p \geq \frac{1}{M - E[X] + 1} = \frac{1}{\binom{k}{2} - \frac{mk(k-1)}{n(n-1)} + 1}.\]
      The expected number of attempts until a suitable $X$ is found is therefore $\frac{1}{p}$ (geometric distribution), whereby the expected runtime is polynomial in $n$, $m$, and $k$.
    \end{proof}

  \item We can use a very similar algorithm to the above, this time randomly assigning vertices to create a cut $(A, A^c)$.

    \begin{algorithm}
      \begin{algorithmic}
        \Function{alg}{$G$}
        \Loop
          \For{$v \in V$}
            \State include $v$ in $A$ with probability $\frac{1}{2}$
          \EndFor
          \If{$|(A, A^c)| \geq \frac{m}{2}$}
            \State \Return $(A, A^c)$
          \EndIf
        \EndLoop
        \EndFunction
      \end{algorithmic}
    \end{algorithm}

    Similarly to (1), this algorithm is correct if it terminates, because it only returns once it has verified its output has the required property.

    {\bf Claim 1:} The expected size of the cut found by a given iteration of $\alg$ is $\frac{m}{2}$.

    \begin{proof}
      For each edge of $G$, define a random variable \[X_{(u,v)} = \begin{cases}1 & u \in A, v \in A^c\\ 1 & u \in A^c, v \in A\\ 0 & \textnormal{otherwise}.\end{cases}\]
      Thus, for a given $(u,v) \in E(G)$, \[E[X_{(u,v)}] = P(u \in A)P(v \in A^c \mid u \in A) + P(u \in A^c)P(v \in A \mid u \in A^c) = \frac{1}{4} + \frac{1}{4} = \frac{1}{2}.\]
      Then \[E\left[|(A, A^c)|\right] = E\left[\sum_{e \in E(G)} X_e\right] = \sum_{e \in E(G)}E[X_e] = mE[X_e] = \frac{m}{2}.\]
    \end{proof}

    {\bf Claim 2:} $\alg$ has runtime polynomial in $n$ and $m$.
    \begin{proof}
      The analysis is very similar to that for (1).
      We again have \[p = P(X \geq E[X]) \geq \frac{1}{M - E[X] + 1}\] where $X$ is the size of the candidate cut and $M$ is the maximum possible value of $X$.
      In this case, $E[X] = \frac{1}{2}$ and $M = m$, so we have \[p \geq \frac{2}{m+2}.\]
      Therefore the expected number of attempts until $\alg$ finds a suitable cut is $\frac{1}{2}(m+2)$.
      The runtime of each attempt is linear in $n$ and $m$, since creating the candidate cut requires a single iteration through $V(G)$ and checking its size requires $\Theta(m)$ time, since we can do so in a single scan of $E(G)$.
      Thus, the total expected runtime is polynomial in $m$ and $n$.
    \end{proof}
\end{enumerate}

\newpage

\section{Solution to Question 4}

\textbf{First Method:}

Consider $h(a) - h(b) = [f_a(r) - f_b(r)]$ mod $p$ \\
If $a \neq b$ but, $h(a) - h(b) = 0$ then the algorithm will return false negative(incorrect result). Thus, we will try to determine
the probability that algorithm fails $i.e.$ it returns false negative.

Now, consider $h(a) - h(b) = [\sum_{i=0}^{n-1} (a_i-b_i) x^i]$ mod $p$ under following assumption: \\
\textbf{Assumption 1:} p  is very large such that the maximum value of $h(a) - h(b)$ is less than p.\\
\\
Under the above assumption, if $ a \neq b$ and $h(a) - h(b) = [f_a(r) - f_b(r)]$ mod $p= 0$, it is only possible when
$f_a(r) - f_b(r) = 0$ since p is a prime number. \\
$i.e.$ $(a_{n-1} - b_{n-1}) x^{n-1} + (a_{n-2} - b_{n-2}) x^{n-2} + ......+ (a_0 - b_0)1 = 0$ \\
The maximum degree of this polynomial is $n-1$, when $a_{n-1} \neq b_{n-1}$.\\
Thus this polynomial can have at most $n-1$ number of roots. \\
Using Schwartz-Zippel algorithm, for any $r$ chosen randomly from a set of t values $(1,2,3,......,t)$ \\
$P(h(a) - h(b) = 0$ when $a \neq b) \leq (n-1)/t$ \\
$P(h(a) \neq h(b) \mid a \neq b) = 1 - P(h(a) = h(b) \mid a \neq b)$ \\
$P(h(a) \neq h(b) \mid a \neq b) \geq 1- (n-1)/t  \geq 1 - 1/n$ \\
Hence, smallest possible values of $t = n(n-1)$ \\

The above t is only true when p is very large(from assumption 1). Now, let's determine the smallest possible value for p.\\
The maximum possible value of $f_a(r) - f_b(r) = \sum_{i=0}^{n-1} (a_i-b_i) r^i$ is when each $a_i - b_i$ is maximum positive and
$r$ is maximum. \\
Since, $a$ and $b = (0,1,2...,m-1)$, the maximum value of $a_i - b_i = m-1$ when $a = m-1$ and $b= 0$.\\
Also, the maximum possible value for $r = (1,2,3,......,t)$ is $t = n(n-1)$. \\
Thus, the maximum possible value for  $f_a(r) - f_b(r) = \sum_{i=0}^{n-1} (m-1)(t^i)$ where $t = n(n-1)$.\\
$ = (m-1)(t^n - 1)/(t-1)$ \\
So, the smallest value of p should be the first prime number greater than\\ $(m-1)(({n(n-1))}^n - 1)/(n(n-1)-1)$.

\textbf{Claim 1:} These are the smallest possible values for both t and p. 
\begin{proof}
	\textbf{Case 1:} If t is smaller than the above value, let us suppose $t = (n-1)(n-1)$, then using Schwartz-Zippel theorem \\
	$P(h(a) = h(b)$ when $a \neq b) \leq (n-1)/((n-1)(n-1))$ \\
	$\Rightarrow P(h(a) \neq h(b) \mid a \neq b) \geq 1 - 1/(n-1)$  which is less than the required probability $1 - 1/n$ \\
	\textbf{Case 2:} If p is smaller than the above value, then there is a possibility that for some value of r 
	$f_a(r) - f_b(r) = p$ and thus $h(a) - h(b) = 0$ since p mod p $= 0$\\
	$i.e. f_a(r) - f_b(r) - p = 0$ which is again a polynomial of degree at most $n-1$. Using Schwartz-Zippel, \\
	$P(h(a) = h(b)$ when $a \neq b) = P(f_a(r) - f_b(r) = 0 \mid a \neq b) + P(f_a(r) - f_b(r) = p \mid a \neq b) \leq 2(n-1)/t$
	$\Rightarrow P(h(a) \neq h(b) \mid a \neq b) \geq 1 - 2(n-1)/t \geq 1 - 1/n$ \\
	Thus, $t = 2n(n-1)$ which is more than  the smallest determined value $n(n-1)$ which we do not want. \\
	Hence the above claim is valid.
\end{proof}

\textbf{Second Method:}
To find the smallest possible value of r, we will use the following result: \\

\textbf{Result 1:} Consider two strings: $a = a_{n-1}a_{n-2}.....a_2a_1a_0$ and $a = b_{n-1}b_{n-2}.....b_2b_1b_0$ such that $a_{n-1} 
\neq b_{n-1}$
while  remaining $a_i's$ and $b_i's$ can be anything(equal or non-equal). \\
Now, for any given r, $f_a(r) - f_b(r) =  \sum_{i=0}^{n-1} (a_i-b_i)r^i$ \\
The minimum value of this difference, given that $a_{n-1} \neq b_{n-1}$ is $ = r^{n-1}$ \\
when $a_{n-1} - b_{n-1} = 1$ ($\neq 0$ since they are different) and for remaining $i's$ from $0$ to $n-2$ $a_i$ = $b_i$. \\
Now,  without changing $b_{n-1}$  if we can find some combination of other $b_i's$ such that $a \neq b$ and $f_a(r) = f_b(r)$ \\
then for those $a, b(new)$, we will always get $h(a) = h(b)$ regardless of any p. \\
To avoid this, we need to determine a well conditioned r such that this is not possible. \\
As calculated above, minimum possible value of difference, $diff_{min}$ $f_a(r) - f_b(r) = r^{n-1}$ \\
while maximum possible value we can add to $f_b(r)$ by changing $(b_{n-2}....b_0)$, $add_{max} = \sum_{i=0}^{n-2}(m-1)r^i$ \\

If $diff_{min} > add_{max}$ then there is no way that we can have $f_a(r) = f_b(r)$ given $a_{n-1} \neq b_{n-1}$ \\
So, $r^{n-1} > (m-1) \sum_{i=0}^{n-2}r^i$ \\
$\Rightarrow (m-r)r^{n-1} < m-1$ \\
which is only true when $r \geq m$ assuming $n > 1$ and $m > 1$.\\
So, smallest possible value of $r = m$.

\textbf{Note} that the above result is valid for all possible strings irrespective of the fact that ${n-1}^{th}$ index is equal or not 
in a and b. This can be implied using the following: \\
Let say first non-matching value of a and b  occurs at index k. Then we need to compensate for the deficit  $(a_k - b_k) r^k$ for 
$f_a(r) - f_b(r) = 0$ by changing $b_i$: \\
Case 1: We only change $b_i$ where $i = (0 to k-1)$. In this case we  can never compensate for the deficit, given the above result.\\
Case 2: We decide to change $b_i$ where $ i \in (k to n-1)$ then we will create a new deficit $(a_i-b_i)r^i$ which again cannot be 
compensated by changing any of the following indices in b given above result.

To determine smallest possible value of t, we follow the same procedure as done in analysis of first method.\\
maximum possible value of $f_a(r) - f_b(r) = f_a(m) - f_b(m)$ \\
$(f_a(m) - f_b(m))_{max} = (m-1) \sum_{i=0}^{n-1} m^i$ \\
$ \Rightarrow  (f_a(m) - f_b(m))_{max} = m^n-1$ \\
Now, we know that the number of primes in $(1, m^n) = (m^n)/ln(m^n)$ by prime numbers theorem.\\
So, any number in $(1, m^n)$ can have at most these many  number of primes as the next prime which we can include will make
it larger than the maximum range $(m^n-1)$.\\
let $x =  (m^n)/ln(m^n)$
So, for some $p$ chosen randomly from a set of first t prime values $(p_1,p_2,p_3,......,p_t)$ \\
$P(h(a) - h(b) = 0$ when $a \neq b) \leq x/t$ because any difference in the possible range can have maximum x number of primes
and will result in 0 when mod p for x number of values of p.\\
$P(h(a) \neq h(b) \mid a \neq b) = 1 - P(h(a) = h(b) \mid a \neq b)$ \\
$P(h(a) \neq h(b) \mid a \neq b) \geq 1- x/t  \geq 1 - 1/n$ \\
Hence, smallest possible values of $t = nx = n(m^n)/ln(m^n)$ \\

\newpage
\section{Solution to Question 5}



\end{document}
