\documentclass[11pt]{article}
\usepackage{fullpage} % better margins
\usepackage{amsthm,amssymb,amsmath} % useful math commands
\usepackage{graphicx} % figures
\usepackage{xfrac} % inline fractions
\usepackage{enumerate} % custom enum symbols
\usepackage{algpseudocode,algorithm} % for pseudocode

% Define some useful shortcuts
\newcommand{\R}{\mathbb{R}}
\newcommand{\C}{\mathbb{C}}
\newcommand{\N}{\mathbb{N}}
\newcommand{\Z}{\mathbb{Z}}
\newcommand{\Q}{\mathbb{Q}}
\newcommand{\M}{\mathcal{M}}
\newcommand{\I}{\mathcal{I}}
\newcommand{\E}{\mathcal{E}}
\renewcommand{\O}{\mathcal{O}}
% if you want to use \alg and \opt in text, wrap them in math mode to make the spacing work
% e.g. We have shown that $\alg$ runs in $\O(n)$
\newcommand{\alg}{\textsc{alg}}
\newcommand{\opt}{\textsc{opt}}

\newtheorem{theorem}{Theorem}[section]
\newtheorem{corollary}{Corollary}[theorem]
\newtheorem{lemma}[theorem]{Lemma}

\title{Solutions for 611 Homework 3}

\author{Mahim Agarwal, Hanwen Xiong, and Jackson Warley}

\begin{document}

\maketitle

\section{Solution to Question 1}



\section{Solution to Question 2}

Our algorithm is as follows.

\begin{algorithm}
  \begin{algorithmic}
    \Function{alg}{$G = (V, E)$}
      \State run Floyd-Warshall on $G$
      \State let $D$ be the resulting table
      \For{$(u, v) \in E$}
        \If{$w(u, v) > D[u, v]$}
          \State set $w(u, v) = D[u, v]$
        \EndIf
      \EndFor
      \State \Return $G$ with the updated weights
    \EndFunction
  \end{algorithmic}
\end{algorithm}

\noindent {\bf Claim 1:} $\alg$ terminates in polynomial time.
\begin{proof}
  Since Floyd-Warshall always terminates, and nothing can cause $\alg$ to consider an edge more than once, $\alg$ will terminate after it has inspected every edge.
  The single run of Floyd-Warshall requires $\O(|V|^3)$ time, as we showed in class.
  The remaining work to be done is a single scan through the edges of the graph, with the constant time work of updating $w(e)$ done for each edge.
  Thus the runtime of $\alg$ is $\O(|V|^3 + |E|) = \O(|V|^3)$ since $|E| = \O(|V|^2)$.
\end{proof}

\noindent {\bf Claim 2:} $\alg$ produces a graph in which every edge satisfies the triangle inequality.
\begin{proof}
  We call an edge $e$ {\em bad\/} in $G$ if it violates the triangle inequality in $G$.
  Otherwise, we say $e$ is {\em good}.

  Firstly, any time an edge weight is reduced by $\alg$, the number of bad edges decreases by one.
  As proof, suppose that on some iteration of the loop, $e = (u, v)$ has its weight reduced from $w_{old}$ to $D[u, v]$.
  Since $D[u, v]$ is the shortest path between $u$ and $v$ by definition, $e$ is no longer bad, so $G$ has lost one bad edge.
  However, it could be that reducing $w(e)$ caused some previously good edge $e' = (u', v')$ to become bad.
  To see that this is not the case, first note that the only way reducing $w(e)$ could have affected $\delta(u', v')$ is if the shortest path $P$ between $u'$ and $v'$ used $e$.
  If this was the case, then $P$ originally incurred the distance $w_{old}$ to travel from $u$ to $v$ (or vice versa).
  But we reduced $w(e)$ to $D[u, v] < w_{old}$, meaning that there was already a path $Q$ of length $D[u, v]$ between $u$ and $v$, which would contradict the assumption that $P$ was the shortest path between $u'$ and $v'$ (since $P$ could have taken $Q$ instead of $e$ and been shorter).
  Thus, decreasing $w(e)$ removed one bad edge from the graph and preserved the number of good edges.

  Secondly, note that any time $\alg$ examines an edge but does not reduce its weight, the number of bad edges is unchanged, since no modifications were made to the graph.
  Since every edge is either good or bad and every edge is either reduced or not reduced by $\alg$, we may now conclude that the number of bad edges in $G$ never increases, and it decreases by one for each bad edge in $G$ -- in other words, there are no bad edges remaining after $\alg$ terminates.
\end{proof}

\section{Solution to Question 3}



\section{Solution to Question 4}



\section{Solution to Question 5} 



\section{Solution to Question 6}

Let $V' = V \cup \{s, t\}$, $E' = E \cup \{(s, u)\}_{u \in L} \cup \{(v, t)\}_{v \in R}$, and $G' = {(V', E')}$.
Assign a direction to each edge in $E'$ such that $s$ is a source, $t$ is a sink, and edges in $E$ are directed from $L$ to $R$.
Finally, give every edge unit capacity for the purpose of defining a flow on $G'$.
We show that the following are equivalent:

\begin{enumerate}
  \item There exists a perfect matching in $G$.
  \item A maximum flow $f$ on $G'$ has size $n$.
  \item A minimum cut $C$ on $G'$ has capacity $n$.
  \item For all $S \subseteq L$, $|\Gamma(S)| \geq |S|$.
\end{enumerate}

\begin{proof}
  \noindent $(1 \iff 2)$ Suppose $G$ has a perfect matching $M$.
  Define the flow \[f(u, v) = \begin{cases}
    1 & \textnormal{if $(u,v) \in M$}\\
    1 & \textnormal{if $u = s$ and $v \in L$}\\
    1 & \textnormal{if $v = t$ and $u \in R$}\\
    -1 & \textnormal{if $f(v, u) = 1$}\\
    0 & \textnormal{otherwise} \end{cases}\] on $G'$.
  It is immediate from the definition that $f$ is skew-symmetric.
  Since $c_{uv} = 1$ for all edges $(u, v) \in E'$, $f_{uv} \leq c_{uv}$.
  For each $u \in L$, $u$ has a single incoming unit of flow, since there is exactly one edge from $s$ to $u$ and no edges from $t$ or $R$ to $u$ by construction, and no edges from $L$ to $u$ because $G$ is bipartite.
  Since $M$ is perfect, there is exactly one $v \in R$ adjacent to $u$, and by construction $u$ is emitting one unit of flow along $(u, v)$.
  A similar argument applies to vertices in $R$, concluding the proof that $f$ is indeed a flow.
  Now, we have \begin{align*}
    |f| &= \sum_{v \in V} f(s, v)\\
    &= \sum_{v \in L} f(s, v) + \sum_{v \in R} f(s,v) + f(s, s) + f(s, t)\\
    &= n + 0 + 0 + 0.
  \end{align*}
  Finally, note that $f$ is maximum, since there are only $n$ edges of unit capacity exiting $s$, which concludes the forward direction.

  For the other direction, suppose that some maximum flow $f$ on $G'$ has size $n$.
  Since $t$ has indegree $n$, and each incoming edge has unit capacity, it must be the case that every edge into $t$ carries unit flow, and equivalently, each $v \in R$ emits unit flow.
  By conservation of flow, each $v \in R$ must receive unit flow, as well.
  Since the edges out of $s$ have unit capacity, no vertex in $L$ receives more than unit flow.
  Thus no two vertices in $R$ recieve flow from the same vertex in $L$.
  Since our flow is integral, no two vertices in $L$ can supply the same vertex in $R$ with flow.
  Thus we have constructed a bijection between the vertices in $R$ and those in $L$, which exactly identifies a perfect matching.
\end{proof}

\begin{proof}
  $(2 \iff 3)$ This is the max-flow min-cut theorem.
\end{proof}

\begin{proof}
  $(3 \iff 4)$ For the forward direction, suppose a minimum cut has capacity $n$.
  Let $S \subseteq L$.
  Then $(A, B) = (S \cup \{s\}, (L \setminus S) \cup R \cup \{t\})$ is a cut, so its capacity is at least $n$.
  Let $K$ be the set of edges from $A$ to $B$.
  Every vertex in $L$ is either in $S$ or not.
  If so, then its outgoing edges are in $K$; if not, then its incoming edges are in $K$.
  By construction, $K$ only includes edges incident to vertices in $L$, so because every edge has unit capacity, we have the identity \[|K| = C(A, B) = |\Gamma(S)| + |L \setminus S|\] for any $S \subseteq L$.
  Thus, we have \[|\Gamma(S)| = C(A, B) - |L \setminus S| \geq n - |L \setminus S| = |L| - |L \setminus S| = |S|\]
  as desired.

  For the backward direction, suppose $|\Gamma(S)| \geq |S|$ for all $S \subseteq L$.
  Let $(A, B)$ be any $s-t$ cut, and let $S = A \cap L$.

\end{proof}

\end{document}
